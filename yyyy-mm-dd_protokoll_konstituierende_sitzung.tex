\documentclass[sitzung=fsv-konstituierend]{fsphys-protokoll}

%%% Datum, Uhrzeit, Teilnehmer, Ort
\renewcommand*{\protokolldatum}{yyyy-mm-dd}
\renewcommand*{\protokollbeginn}{hh:mm~Uhr}
\renewcommand*{\protokollende}{hh:mm~Uhr}
\renewcommand*{\protokollant}{Name des Protokollanten}
\renewcommand*{\protokollort}{StudiO, Wilhelm-Klemm-Str.~9, 48149~Münster}
% Listen unbedingt durch Kommata trennen, sonst wird nicht richtig gezählt.
% Anwesende Mitglieder der FSV
\renewcommand*{\protokollanwesend}{Namen}
% Weitere Anwesende auf der Sitzung
\renewcommand*{\protokollweitere}{Namen}

\begin{document}
	
\section{Begrüßung und Wahl einer Versammlungsleitung und eines Protokollanten}
\begin{itemize}
	\item XXX begrüßt alle Anwesenden.
	\item YYY wird mit 00J:00N:00E\footnote{Hinweis zur Konvention: Ergebnisse von Abstimmungen werden in der Reihenfolge (Ja-Stimmen:Nein-Stimmen:Enthaltungen) angegeben.} zur Versammlungsleitung gewählt.
	\item ZZZ wird mit 00J:00N:00E zum Protokollanten/zur Protokollantin gewählt.
\end{itemize}

\section{Feststellung der Beschlussfähigkeit}
Für die Beschlussfähigkeit müssen mindestens \SI{50}{\percent} der wahlberechtigten Mitglieder anwesend sein.

Die Beschlussfähigkeit wird mit XX anwesenden von insgesamt 15~Mitgliedern der Fachschaftsvertretung festgestellt.

\section{Feststellung der Tagesordnung}
Festgestellt wie vorgestellt.

\section{Wahl des neuen Fachschaftsrats}
\begin{longtable}{| r @{ } l | l | c |}
	\hline
	& "Amt" (kann entfallen)
	& \parbox[t]{5cm}{\centering Vor- und Nachname;\\ Matrikel-Nr.}
	& \parbox[t]{2.8cm}{\centering Abstimmung\\(Ja:Nein:Enth.)}
	\\ \hline\hline
	\endhead
	1.  & Vorsitzende/r
		& &
	\\ \hline
	2.  & Stellv.\ Vorsitzende/r
		& &
	\\ \hline
	3.  & Finanzreferent/in
		& &
	\\ \hline
	4.  & Stellv.\ Finanzreferent/in
		& &
	\\ \hline
	5.  & Evaluationsbeauftragte/r
		& &
	\\ \hline
	6.  & Stellv.\ Evaluationsbeauftragte/r
		& &
	\\ \hline
	7.  & Stellv.\ Evaluationsbeauftragte/r
		& &
	\\ \hline
	8.  & Ersti-Beauftragte/r
		& &
	\\ \hline
	9.  & Stellv.\ Ersti-Beauftragte/r
		& &
	\\ \hline
	10. & Beauftragte/r O-Woche
		& &
	\\ \hline
	11. & Stellv.\ Beauftragte/r O-Woche
		& &
	\\ \hline
	12. & Beauftragte/r Ersti-Fahrt
		& &
	\\ \hline
	13. & Stellv.\ Beauftragte/r Ersti-Fahrt
		& &
	\\ \hline
	14. & Beauftragte/r Ersti-Fibel
		& &
	\\ \hline
	15. & Stellv.\ Beauftragte/r Ersti-Fibel
		& &
	\\ \hline
	16. & Beauftragte/r Sommerfest
		& &
	\\ \hline
	17. & Stellv.\ Beauftragte/r Sommerfest
		& &
	\\ \hline
	18. & Stellv.\ Beauftragte/r Sommerfest
		& &
	\\ \hline
	19. & Eventmanager/in
		& &
	\\ \hline
	20. & Stellv.\ Eventmanager/in
		& &
	\\ \hline
	21. & Beauftragte/r BaMa-Tag
		& &
	\\ \hline
	22. & Stellv.\ Beauftragte/r BaMa-Tag
		& &
	\\ \hline
	23. & Beauftragte/r Buchmarkt
		& &
	\\ \hline
	24. & Stellv.\ Beauftragte/r Buchmarkt
		& &
	\\ \hline
	25. & Beauftragte/r Spieleabend
		& &
	\\ \hline
	26. & Stellv.\ Beauftragte/r Spieleabend
		& &
	\\ \hline
	27. & Beauftragte/r Protokoll-/Klausurverleih
		& &
	\\ \hline
	28. & Stellv.\ Beauftr.\ Protokoll-/Klausurverleih
		& &
	\\ \hline
	29. & Lehrpreis-Komitee
		& &
	\\ \hline
	30. & Lehrpreis-Komitee
		& &
	\\ \hline
	31. & Lehrpreis-Komitee
		& &
	\\ \hline
	32. & Beauftragte/r Öffentlichkeitsarbeit
		& &
	\\ \hline
	33. & Stellv.\ Beauftr.\ Öffentlichkeitsarbeit
		& &
	\\ \hline
	34. & E-Mail-Beauftragte/r
		& &
	\\ \hline
	35. & Stellv.\ E-Mail-Beauftragte/r
		& &
	\\ \hline
	36. & Imperia-Beauftragte/r
		& &
	\\ \hline
	37. & Stellv.\ Imperia-Beauftragte/r
		& &
	\\ \hline
	38. & PC-Administrator/in
		& &
	\\ \hline
	39. & PC-Administrator/in
		& &
	\\ \hline
	40. & Beauftragte/r Hochschultag/Schüler
		& &
	\\ \hline
	41. & Stellv.\ Beauftr.\ Hochschultag/Schüler
		& &
	\\ \hline
\end{longtable}

\clearpage

\appendix

\section{Mitgliederliste des Fachschaftsrats}
\begin{longtable}{| r @{ } l | >{\raggedright\arraybackslash}m{10cm} |}
	\hline
	& Vor- und Nachname & "Ämter"
	\\ \hline\hline
	\endhead
	1.  & 
	&
	\\ \hline
	2.  & 
	&
	\\ \hline
	3.  & 
	&
	\\ \hline
	4.  & 
	&
	\\ \hline
	5.  & 
	&
	\\ \hline
	6.  & 
	&
	\\ \hline
	7.  & 
	&
	\\ \hline
	8.  & 
	&
	\\ \hline
	9.  & 
	&
	\\ \hline
	10. & 
	&
	\\ \hline
	11. & 
	&
	\\ \hline
	12. & 
	&
	\\ \hline
	13. & 
	&
	\\ \hline
	14. & 
	&
	\\ \hline
	15. & 
	&
	\\ \hline
	16. & 
	&
	\\ \hline
	17. & 
	&
	\\ \hline
	18. & 
	&
	\\ \hline
	19. & 
	&
	\\ \hline
	20. & 
	&
	\\ \hline
\end{longtable}

\bigskip
\bigskip
Das Protokoll wurde genehmigt auf der Sitzung am \protokollformatteddate.

\vspace{2cm}
Unterschrift des Protokollführers/der Protokollführerin

\end{document}
