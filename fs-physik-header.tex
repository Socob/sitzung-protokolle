%%%%%%%%%%%%%%%% Pakete & Klassen %%%%%%%%%%%%%%%%%%%%%%%%%%%%%%%%%%%%%%%%%%%%%%%%%%%%%%%%%%%%%%%%%%%%%%%%%%%%%%%%%%%%%%%%%%%%%%%%%%%%
\documentclass[a4paper, 12pt, oneside]{article}             % Papierformat, Schriftgröße, einseitiges Layout; Verwendung des KOMA-Script
%\usepackage[ansinew, latin1]{inputenc}                     % Zeichenkodierung: Übersetzung von Umlauten, Windows/Unix-Umgebung; deutsche Umlaute
\usepackage[utf8]{inputenc}                                 % Zeichenkodierung
\usepackage[T1]{fontenc}                                    % Schriftart
\usepackage[ngerman]{babel}                                 % neue deutsche Rechtscheibung, Silbentrennung
\usepackage{csquotes}                                       % "Schlaue" Anführungszeichen
\usepackage{lmodern}                                        % Schriftart
\usepackage{textcomp}                                       % führt Sonderzeichen ein
\usepackage{url}                                            % URLs sind Hyperlinks
\usepackage{hyperref}                                       % Verlinkung, jeder Querverweis kann im PDF angeklickt werden
\usepackage{color}                                          % Farben ermöglichen. Grundfarben: black, white, red, green, yellow, blue
\usepackage{longtable}                                      % für extra lange Tabellen
\usepackage{enumitem}                                       % schönere Aufzählungen
\usepackage{multirow}                                       % \multirow{} und \multicolumn{} für Tabellen
\usepackage{graphicx}                                       % erweitertes Paket für Bilder-Einbindung (EPS, PNG, JPG, PDF)
\usepackage[left=1.8cm,right=1.8cm,top=1.8cm]{geometry}     % Seitenränder
\usepackage[final]{pdfpages}				                % Einfügen von mehrseitigen PDF-Dateien,
						  % benutzen mit \includepdf[pages=-, pagecommand={\thispagestyle{headings}}]{doc.pdf}
\pagestyle{myheadings}                                      % ermöglicht Kopfzeilen
\MakeOuterQuote{"}                                          % Anführungszeichen automatisch umwandeln


%%%%%%%% Sachen definieren %%%%%%%%%%%%%%%%%%%%%%%%%%%%%%%%%%%%%%%%%%%%%%%%%%%%%%%%%%%%%%%%%%%%%%%%%%%%%%%%%%%%%

\makeatletter
\newcounter{protokollcounter}
\newcommand{\protokollparse}[1]{\@for\@tempa:=#1\do{\stepcounter{protokollcounter}}}

\newcommand{\protokolldatum}{00:00:0000}
\newcommand{\protokollbeginn}{00:00 Uhr}
\newcommand{\protokollende}{00:00 Uhr}
\newcommand{\protokollant}{NN}
\newcommand{\protokollanwesend}{NN}
\newcommand{\protokollfehlend}{NN}
\newcommand{\cntprotokollanwesend}{\setcounter{protokollcounter}{0}\protokollparse{\protokollanwesend}\arabic{protokollcounter}}
\newcommand{\cntprotokollfehlend}{\setcounter{protokollcounter}{0}\protokollparse{\protokollfehlend}\arabic{protokollcounter}}

\newcommand{\protokolltitlepage}{%
  \begin{center}
        \includegraphics[width=0.75\textwidth]{fs-physik-logo.pdf}\\
        \vspace{0.3cm}
        \Huge{\textbf{Protokoll der Fachschaftssitzung} \\
        \vspace{0.2cm}
        \protokolldatum\\
        \vspace{0.2cm}
        }
        \normalsize

		\begin{longtable}{|p{6.1cm}|p{6.1cm}|}
			\hline
			\textbf{Beginn:}    \protokollbeginn &
			\textbf{Ende:}      \protokollende \\
			\hline
			\textbf{Protokollant:} &
			\protokollant \\
			\hline
			\multicolumn{2}{|p{12.5cm}|}{\textbf{Anwesende (\cntprotokollanwesend): }} \\
			\multicolumn{2}{|p{12.5cm}|}{\protokollanwesend}\\
			\hline
			\multicolumn{2}{|p{12.5cm}|}{\textbf{Entschuldigt (\cntprotokollfehlend): }} \\
			\multicolumn{2}{|p{12.5cm}|}{\protokollfehlend}\\
			\hline
		\end{longtable}
	\end{center}
	\tableofcontents
}
