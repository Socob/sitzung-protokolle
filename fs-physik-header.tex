% Header-Datei für Protokolle der Fachschaft Physik
% 2014-04-17, Simon May
%%% Pakete & Klassen %%%
\documentclass[
  a4paper,          % Papierformat
  12pt,             % Schriftgröße
  oneside,          % einseitiges Layout
  ngerman           % neue deutsche Rechtschreibung
]{scrartcl}         % Verwendung des KOMA-Script
\usepackage{babel}                % Silbentrennung, sprachspezifische Einstellungen
\usepackage[T1]{fontenc}          % Mögliche darstellbare Zeichen (Umlaute, Sonderzeichen...)
\usepackage[utf8]{inputenc}       % Zeichenkodierung der TeX-Datei
\usepackage{lmodern}              % Schriftart
\usepackage{textcomp}             % führt Sonderzeichen ein
\usepackage{scrpage2}             % Für \ihead{}, \ifoot{} etc.
\usepackage{xcolor}               % Farben ermöglichen
\usepackage{csquotes}             % "Schlaue" Anführungszeichen
\usepackage[
  bookmarksnumbered, unicode
]
{hyperref}                        % Verlinkung, Querverweise können im PDF angeklickt werden
\usepackage{enumitem}             % schönere Aufzählungen
\usepackage{longtable}            % für extra lange Tabellen
\usepackage{graphicx}             % Paket für Bilder-Einbindung (EPS, PNG, JPG, PDF)
\usepackage[
  left=1.8cm, right=1.8cm, top=1.8cm, bottom=1.5cm
]{geometry}                       % Seitenränder
\usepackage[final]{pdfpages}      % Einfügen von mehrseitigen PDF-Dateien, benutzen mit \includepdf[pages=-, pagecommand={\thispagestyle{headings}}]{doc.pdf}

%%% Einstellungen %%%
% ermöglicht Kopfzeilen
\pagestyle{scrheadings}
% Kopfzeile links
\ihead{Protokoll der Fachschaftssitzung vom \protokolldatum}
% Kopfzeile rechts
\ohead{\pagemark}
% Fußzeile leer
\cfoot{}

% Anführungszeichen automatisch umwandeln
\MakeOuterQuote{"}
% Einstellungen für Aufzählungen
\setlist{itemsep=0cm, parsep=0cm, leftmargin=2em}
% Hintergrundfarbe
\pagecolor{white}
% Textfarbe
\color{black}

%%% Definitionen %%%
\makeatletter
\newcounter{protokollcounter}
\newcommand{\protokollparse}[1]{\@for\@tempa:=#1\do{\stepcounter{protokollcounter}}}

\newcommand{\protokolldatum}{00.00.0000}
\newcommand{\protokollbeginn}{00:00 Uhr}
\newcommand{\protokollende}{00:00 Uhr}
\newcommand{\protokollant}{NN}
\newcommand{\protokollanwesend}{NN}
\newcommand{\protokollfehlend}{NN}
\newcommand{\cntprotokollanwesend}{\setcounter{protokollcounter}{0}\protokollparse{\protokollanwesend}\arabic{protokollcounter}}
\newcommand{\cntprotokollfehlend}{\setcounter{protokollcounter}{0}\protokollparse{\protokollfehlend}\arabic{protokollcounter}}

\newcommand{\protokolltitlepage}{%
	\begin{center}
		\includegraphics[width=0.75\textwidth]{fs-physik-logo.pdf}\\
		\vspace{0.3cm}
		\Huge{\textbf{Protokoll der Fachschaftssitzung} \\
		\vspace{0.2cm}
		\protokolldatum\\
		\vspace{0.2cm}
		}
		\normalsize

		\begin{longtable}{|p{6.1cm}|p{6.1cm}|}
			\hline
			\endfirsthead
			\textbf{Beginn:}    \protokollbeginn &
			\textbf{Ende:}      \protokollende \\
			\hline
			\textbf{Protokollant:} &
			\protokollant \\
			\hline
			\multicolumn{2}{|p{12.5cm}|}{\textbf{Anwesende (\cntprotokollanwesend): }} \\
			\multicolumn{2}{|p{12.5cm}|}{\protokollanwesend}\\
			\hline
			\multicolumn{2}{|p{12.5cm}|}{\textbf{Entschuldigt (\cntprotokollfehlend): }} \\
			\multicolumn{2}{|p{12.5cm}|}{\protokollfehlend}\\
			\hline
		\end{longtable}
	\end{center}
	\tableofcontents
	\newpage
}

