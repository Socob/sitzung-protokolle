% Header-Datei für Protokolle der Fachschaft Physik der WWU Münster
% Datum: 2014-12-09
% Autor: Simon May

%%% Pakete & Klassen
% Verwendung von KOMA-Script
\documentclass[
	% Papierformat
	a4paper,
	% Schriftgröße
	12pt,
	% einseitiges Layout
	oneside,
	% Papiergröße korrekt ins Dokument schreiben
	pagesize,
	% deutsches Dokument (neue deutsche Rechtschreibung)
	ngerman
]{scrartcl}

% Unterscheidung zwischen pdfTeX und LuaTeX
\usepackage{iftex}
\ifLuaTeX
	% Einstellung der Schriftart
	\usepackage{fontspec}
	\defaultfontfeatures{Ligatures=TeX}
	% Silbentrennung, sprachspezifische Einstellungen
	\usepackage{polyglossia}
	\setmainlanguage{german}
\else
	% Silbentrennung, sprachspezifische Einstellungen
	\usepackage{babel}
	% Mögliche darstellbare Zeichen (Umlaute, Sonderzeichen...)
	\usepackage[T1]{fontenc}
	% Zeichenkodierung der TeX-Datei
	\usepackage[utf8]{inputenc}
	% Schriftart
	\usepackage{lmodern}
	% führt Befehle für Sonderzeichen ein
	\usepackage{textcomp}
\fi
% Für Kopf-/Fußzeile etc.
\usepackage{scrpage2}
% Farben ermöglichen
\usepackage{xcolor}
% schönere Aufzählungen
\usepackage{enumitem}
% "Schlaue" Anführungszeichen
\usepackage{csquotes}
% für extra lange Tabellen
\usepackage{longtable}
% Paket für Bilder-Einbindung (EPS, PNG, JPG, PDF)
\usepackage{graphicx}
% Zum Einbinden von Zitaten oder Textdateien
\usepackage{listingsutf8}
\usepackage{euro}
% Zur einfachen Eingabe von Zahlen mit Einheiten
\usepackage{siunitx}
% Seitenränder
\usepackage[
	left=1.8cm,
	right=1.8cm,
	top=1.8cm,
	bottom=1.5cm
]{geometry}
% Verlinkung, Querverweise können im PDF angeklickt werden
\usepackage[bookmarksnumbered, unicode]{hyperref}

%%% Einstellungen
% Mehr Freiraum in Tabellen
\renewcommand{\arraystretch}{1.1}

% Einstellungen für Sonderzeichen bei listings
\lstset{
	columns=flexible,
	breaklines=true,
	basicstyle=\ttfamily,
	literate=
		{ö}{{\"o}}1
		{ä}{{\"a}}1
		{ü}{{\"u}}1
		{Ö}{{\"O}}1
		{Ä}{{\"A}}1
		{Ü}{{\"U}}1
		{ß}{{\ss}}1
		{§}{{\S}}1
}

% Anführungszeichen automatisch umwandeln
\MakeOuterQuote{"}

% Einstellungen für siunitx
\sisetup{mode=text}

% Einstellungen für Aufzählungen
\setlist{leftmargin=2em}

% Kopfzeile links
\ihead{Protokoll der Fachschaftssitzung vom \protokolldatum}
% Kopfzeile rechts
\ohead{\pagemark}
% Fußzeile leer
\cfoot{}
% Dokument mit Kopf-/Fußzeile
\pagestyle{scrheadings}

%%% Definitionen
\makeatletter
\newcounter{protokollcounter}
\newcommand{\protokollparse}[1]{\@for\@tempa:=#1\do{\stepcounter{protokollcounter}}}

\newcommand{\protokolldatum}{00.00.0000}
\newcommand{\protokollbeginn}{00:00 Uhr}
\newcommand{\protokollende}{00:00 Uhr}
\newcommand{\protokollant}{NN}
\newcommand{\protokollanwesend}{NN}
\newcommand{\protokollfehlend}{NN}
\newcommand{\cntprotokollanwesend}{\setcounter{protokollcounter}{0}\protokollparse{\protokollanwesend}\arabic{protokollcounter}}
\newcommand{\cntprotokollfehlend}{\setcounter{protokollcounter}{0}\protokollparse{\protokollfehlend}\arabic{protokollcounter}}

\newcommand{\protokolltitlepage}{
	\begin{center}
		\setlength{\parindent}{0cm}
		\includegraphics[width=0.75\textwidth]{fs-physik-logo.pdf}
		
		{\usekomafont{title}\huge
		Protokoll der Fachschaftssitzung
		\par}
		
		\medskip
		{\usekomafont{date}
		\protokolldatum
		\par}
		
		\medskip
		\begin{longtable}{|p{6.1cm}|p{6.1cm}|}
			\hline
			\endfirsthead
			\textbf{Beginn:}    \protokollbeginn &
			\textbf{Ende:}      \protokollende \\
			\hline
			\textbf{Protokollant:} &
			\protokollant \\
			\hline
			\multicolumn{2}{|p{12.5cm}|}{\textbf{Anwesende (\cntprotokollanwesend): }} \\
			\multicolumn{2}{|p{12.5cm}|}{\protokollanwesend} \\
			\hline
			\multicolumn{2}{|p{12.5cm}|}{\textbf{Entschuldigt (\cntprotokollfehlend): }} \\
			\multicolumn{2}{|p{12.5cm}|}{\protokollfehlend} \\
			\hline
		\end{longtable}
	\end{center}

	\tableofcontents
	\clearpage
}

